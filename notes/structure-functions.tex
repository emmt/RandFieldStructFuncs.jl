\documentclass{article}

\usepackage[a4paper,margin=25mm]{geometry}
\usepackage[dvipsnames,hyperref]{xcolor}
\usepackage[unicode=true,
 bookmarks=true,bookmarksnumbered=false,bookmarksopen=false,
 breaklinks=true,pdfborder={0 0 0},backref=page,colorlinks=true,
 linkcolor=NavyBlue,urlcolor=NavyBlue,citecolor=NavyBlue,]{hyperref}
\usepackage{amsmath,amssymb}
\usepackage{mathtools}
\DeclareMathOperator{\Var}{Var}
\DeclareMathOperator{\Cov}{Cov}
\renewcommand*{\Re}{\mathrm{Re}}

%% Parentheses:
%\def\Paren{}\let\Paren\undefined
\DeclarePairedDelimiterX{\Paren}[1]{(}{)}{#1}
\DeclarePairedDelimiterX{\Brace}[1]{\{}{\}}{#1}
\DeclarePairedDelimiterX{\Brack}[1]{[}{]}{#1}
\DeclarePairedDelimiterX{\Abs}[1]{\rvert}{\lvert}{#1}
\DeclarePairedDelimiterX{\Norm}[1]{\lVert}{\rVert}{#1}
\DeclarePairedDelimiterX{\Avg}[1]{\langle}{\rangle}{#1}
\DeclarePairedDelimiterX{\Round}[1]{\lfloor}{\rceil}{#1}
\DeclarePairedDelimiterX{\Floor}[1]{\lfloor}{\rfloor}{#1}
\DeclarePairedDelimiterX{\Ceil}[1]{\lceil}{\rceil}{#1}
\DeclarePairedDelimiterX{\Inner}[1]{\langle}{\rangle}{#1}
\DeclarePairedDelimiterX{\IntRange}[1]{\llbracket}{\rrbracket}{#1}
\DeclarePairedDelimiterX{\Group}[1]{\lgroup}{\rgroup}{#1}
%\DeclarePairedDelimiterX{\FrobNorm}[1]{\lVert}{\rVert_{\Tag{F}}}{#1}
%% A list \List{ELEM}{START}{END}
\DeclarePairedDelimiterXPP{\List}[3]{}{\{}{\}}{_{#2,\ldots,#3}}{#1}

\begin{document}
\title{Field Structure Function and Covariance}
\author {Éric Thiébaut}
\date{1st version: March 10, 2023; last revision: June 21, 2024}
\maketitle

\abstract{This note describes how to express the covariance $\Cov_{\varphi}$ of a Gaussian
  random field $\varphi \sim \mathcal{N}(0, \Cov_{\varphi})$ given the structure function
  $D_{\varphi}$ of $\varphi$ and the standard deviation $\sigma$ of the \emph{piston} over
  a finite support $\mathcal{V}$. The covariance $\Cov_{\varphi}$ is invertible if
  $\sigma \not= 0$. A fast algorithm to compute the empirical structure function of a
  random field is proposed.}

\section{Structure function of a random field}

The structure function is an important tool for studying non-stationary processes with
stationary increments. A random field $\phi(r)$ is said to have stationary increments if
the statistic of the increment $\delta\phi(r,\Delta r) = \phi(r) - \phi(r + \Delta r)$
only depends on $\Delta r$, not on $r$. The field $\phi(r)$ itself may be non-stationary.
The variable $r$ represents some kind of \emph{position} like space and/or time. An
example of field with stationary increments is the random phase delays caused by a
turbulent medium.

\subsection{Definition and properties of the structure function}

The structure function of the random field $\phi(r)$ with stationary increments is defined
by:
\begin{equation}
  \label{eq:D-phi}
  D_{\phi}(\Delta r) = \Avg*{(\phi(r) - \phi(r + \Delta r))^{2}}
\end{equation}
which holds whatever $r$, since $\phi(r)$ has stationary increments, and where
$\Avg*{\ldots}$ denotes expectation. Structure functions with higher orders than 2
may also be introduced.

Directly following from the definition of the structure function:
\begin{equation}
  D_{\phi}(0) = 0.
\end{equation}

Since Eq.~\eqref{eq:D-phi} holds whatever $r$, the structure function is an even function:
\begin{equation}
  \label{eq:D-phi-even}
  D_{\phi}(-\Delta r) =  D_{\phi}(\Delta r).
\end{equation}

If the mean $\Avg*{\phi(r)}$ does not depend on $r$, then:
\begin{displaymath}
  \phi(r) - \phi(r + \Delta r)
  = [\phi(r) - \Avg*{\phi(r)}] - [\phi(r + \Delta r) - \Avg*{\phi(r + \Delta r)}]
\end{displaymath}
and developping the expression of the field structure function yields:
\begin{align}
  D_{\phi}(\Delta r)
  &= \Avg*{\Paren*{[\phi(r) - \Avg*{\phi(r)}] - [\phi(r + \Delta r) - \Avg*{\phi(r + \Delta r)}]}^{2}} \notag \\
  &= \Avg*{[\phi(r) - \Avg*{\phi(r)}]^{2}} + \Avg*{[\phi(r + \Delta r) - \Avg*{\phi(r + \Delta r)}]^{2}}\notag\\
  &\quad  - 2\,\Avg*{[\phi(r) - \Avg*{\phi(r)}]\,[\phi(r + \Delta r) - \Avg*{\phi(r + \Delta r)}]}\notag \\
  &= \Var_{\phi}(r) + \Var_{\phi}(r + \Delta r) - 2 \Cov_{\phi}(r, r + \Delta r)
  \label{eq:D-phi-given-Cov}
\end{align}
where $\Cov_{\phi}(r,r')$ is the covariance of the field given by:
\begin{align}
  \Cov_{\phi}(r,r')
  &= \Avg*{[\phi(r) - \Avg*{\phi(r)}]\,[\phi(r') - \Avg*{\phi(r')}]}
\end{align}
while $\Var_{\phi}(r) = \Cov_{\phi}(r,r)$ is the variance of the field.

Hence the spatial covariance of the field can be expressed as:
\begin{equation}
  \label{eq:Cov-phi}
  \Cov_{\phi}(r,r') = \frac12\,\Brack*{\Var_{\phi}(r) + \Var_{\phi}(r') - D_{\phi}(r - r')}.
\end{equation}
Properties~\eqref{eq:D-phi-given-Cov} and \eqref{eq:Cov-phi} hold for any random field
$\phi(r)$ with stationary increments and uniform mean, i.e. $\Avg*{\phi(r)}$ is a
constant.


\section{De-Pistoned field}
\label{sec:de-pistoned-field}

The de-pistoned field $\psi(r)$ over a finite size volume $\mathcal{V}$ is defined by
($\forall r \in \mathcal{V}$):
\begin{align}
  \psi(r) = \phi(r) - \frac{1}{|\mathcal{V}|} \int_{\mathcal{V}} \phi(r')\,\mathrm{d}r',
\end{align}
with $|\mathcal{V}|=\int_{\mathcal{V}}\mathrm{d}r$, the volume of $\mathcal{V}$.

Introducing the normalized support function\footnote{so-called normalized \emph{pupil
    function} for turbulent phase screens in optics} $S(r)$ such that for any function
$f(r)$:
\begin{align}
  \label{eq:S-def}
  \int f(r)\,S(r)\,\mathrm{d}r = \frac{1}{|\mathcal{V}|} \int_{\mathcal{V}} f(r)\,\mathrm{d}r,
\end{align}
the de-pistoned field writes:
\begin{align}
  \psi(r) = \phi(r) - \int \phi(r')\,S(r')\,\mathrm{d}r'.
\end{align}

A property of the normalized support function is that:
\begin{align}
  \int S(r)\,\mathrm{d}r = 1,
  \label{eq:int-S=1}
\end{align}
which directly follows from Eq.~\eqref{eq:S-def} by taking $f(r) \equiv 1$.

Since $\psi(r) - \psi(r + \Delta r) = \phi(r) - \phi(r + \Delta r)$, the structure
functions of the field $\phi(r)$ and of the de-pistoned field $\psi(r)$ are the same,
except for the restriction that $\Delta r$ must fit inside $\mathcal{V}$ for the structure
function of the de-pistoned field.

By linearity and since the field $\psi(r)$ has uniform mean and the aperture finite, the
de-pistoned field is centered.
\begin{quote}
  \emph{Proof:}
  \begin{align}
    \Avg*{\psi(r)}
    = \Avg*{\phi(r)} -
      \frac{1}{|\mathcal{V}|} \int_{\mathcal{V}} \Avg*{\phi(r')}\,\mathrm{d}r'
    = \bar{\phi} - \frac{\bar{\phi}}{|\mathcal{V}|} \int_{\mathcal{V}}\mathrm{d}r = 0
  \end{align}
  where $\bar{\phi} = \Avg{\phi(r)}$ does not depend on $r$ and which follows from the
  definition of the volume of $\mathcal{V}$:
  $|\mathcal{V}|=\int_{\mathcal{V}}\mathrm{d}r$. $\blacksquare$
\end{quote}

Since $\Avg*{\psi(r)} = 0$ and since $\bar{\phi} = \Avg{\phi(r)}$ does not depend on $r$,
the covariance of the de-pistoned field can be computed as follows
($\forall (r,r') \in \mathcal{V}^{2}$):
\begin{align}
  \Cov_{\psi}(r,r')
  &= \Avg*{\psi(r)\,\psi(r')} \notag \\
  &= \Avg*{\Paren*{\Brack[\Big]{\phi(r) - \bar{\phi}} - \int \Brack[\Big]{\phi(r'') - \bar{\phi}} \,S(r'')\,\mathrm{d}r''}\,
    \Paren*{\Brack[\Big]{\phi(r') - \bar{\phi}} - \int \Brack[\Big]{\phi(r'') - \bar{\phi}}\,S(r'')\,\mathrm{d}r''}} \notag \\
  &= \Cov_{\phi}(r,r')
    - \int \Cov_{\phi}(r,r'')\,S(r'')\,\mathrm{d}r''
    - \int \Cov_{\phi}(r',r'')\,S(r'')\,\mathrm{d}r'' \notag\\
  &\quad + \iint \Cov_{\phi}(r'',r''')\,S(r'')\,\mathrm{d}r''\,S(r''')\,\mathrm{d}r'''.
    \label{eq:Cov-psi-1}
\end{align}
From Eq.~\eqref{eq:Cov-phi}:
\begin{align}
  \int \Cov_{\phi}(r,r')\,S(r')\,\mathrm{d}r'
  &= \frac{1}{2}\,\int \Paren*{
    \Var_{\phi}(r) + \Var_{\phi}(r') - D_{\phi}(r - r')
    }\,S(r')\,\mathrm{d}r'\notag\\
  &= \frac{1}{2}\,\Var_{\phi}(r)
    + \frac{1}{2}\,\int \Paren*{
    \Var_{\phi}(r') - D_{\phi}(r - r')}\,S(r')\,\mathrm{d}r',
    \label{eq:Cov-psi-2}
\end{align}
and averaging over the aperture one more time yields:
\begin{align}
  \iint \Cov_{\phi}(r,r')\,S(r)\,\mathrm{d}r\,S(r')\,\mathrm{d}r'
  &= \int \Var_{\phi}(r)\,S(r)\,\mathrm{d}r
  - \frac{1}{2}\,\iint D_{\phi}(r - r')\,S(r)\,\mathrm{d}r\,S(r')\,\mathrm{d}r'.
    \label{eq:Cov-psi-3}
\end{align}
Putting Eqs.~\eqref{eq:Cov-phi}, \eqref{eq:Cov-psi-2}, and \eqref{eq:Cov-psi-3}
in Eq.~\eqref{eq:Cov-psi-1} yields:
\begin{align}
  \Cov_{\psi}(r,r')
  &= \frac{1}{2} \int \Paren*{D_{\phi}(r - r'') + D_{\phi}(r' - r'')}\,S(r'')\,\mathrm{d}r''
    - \frac{1}{2} \, D_{\phi}(r - r') \notag\\
  &\quad - \frac{1}{2}\,\iint D_{\phi}(r'' - r''')\,S(r'')\,\mathrm{d}r''\,S(r''')\,\mathrm{d}r'''.
    \label{eq:Cov-psi-4}
\end{align}
Hence the covariance of the de-pistoned field only depends on the structure
function of the field.

To quickly compute $\Cov_{\psi}(r,r')$, it is advantageous to introduce:
\begin{subequations}
  \begin{equation}
    \label{eq:C-phi}
    V_{\psi}(r) = K_{\phi}(r)
    - \frac{1}{2}\,\int K_{\phi}(r')\,S(r')\,\mathrm{d}r',
  \end{equation}
  with:
  \begin{equation}
    \label{eq:K-phi}
    K_{\phi}(r) = \int D_{\phi}(r - r')\,S(r')\,\mathrm{d}r',
  \end{equation}
\end{subequations}
to rewrite the covariance of the de-pistoned field as:
\begin{equation}
  \boxed{
    \Cov_{\psi}(r,r')
    = \frac{1}{2}\,\Brack*{V_{\psi}(r) + V_{\psi}(r') - D_{\phi}(r - r')}.
  }
  \label{eq:Cov-psi}
\end{equation}
Taking $r = r'$ in the above expression straightforwardly shows that
$V_{\psi}(r)$ is the variance $\Var_{\psi}(r)$ of the de-pistoned field $\psi$.
Since the field $\phi$ and the de-pistoned field $\psi$ have the same structure
function Eq.~\eqref{eq:Cov-psi} is the counterpart of Eq.~\eqref{eq:Cov-phi}.


\subsection{Invertible covariance}

Since the piston of the de-pistoned field $\psi(r)$ is zero, the covariance of
the de-pistoned field is rank deficient and thus non-invertible. Adding a
random piston $\alpha \sim \mathcal{N}(0,\sigma^{2})$ to the de-pistoned field
$\psi(r)$ yields the field ($\forall r \in \mathcal{V}$):
\begin{align}
  \label{eq:varphi-def}
  \varphi(r) = \psi(r) + \alpha,
\end{align}
which has an invertible covariance provided $\sigma > 0$. Since $\alpha$ and $\psi(r)$ are
mutually independent whatever $r$, the covariance of the field $\varphi(r)$ is trivial to
compute:
\begin{align}
  \Cov_{\varphi}(r,r')
  &= \Cov_{\psi}(r,r') + \sigma^{2} \notag\\
  &= \frac{1}{2}\,\Brack*{V_{\varphi}(r) + V_{\varphi}(r') - D_{\varphi}(r - r')},
  \label{eq:Cov-varphi}
\end{align}
where $D_{\varphi} = D_{\phi}$ the structure function of $\varphi$ and with:
\begin{equation}
  \label{eq:Var-varphi}
  V_{\varphi}(r) = \Var_{\varphi}(r) = V_{\psi}(r) + \sigma^{2}
\end{equation}
its variance.

The random field $\varphi(r)$ for $r \in \mathcal{V}$, is a centered Gaussian field whose
covariance $\Cov_{\varphi}$ is given by Eqs.~\eqref{eq:Cov-varphi} and
\eqref{eq:Var-varphi} and only depends on the structure function
$D_{\varphi} = D_{\psi} = D_{\phi}$ of the field and on the standard deviation $\sigma$ of
the piston. If $\sigma \not= 0$, the covariance $\Cov_{\varphi}$ is invertible.


\section{Numerical methods}

\subsection{Fast computation of the empirical structure function}

Assume that the field $\phi(r)$ is sampled on a finite size volume
$\mathcal{V}$ and that a set of $T$ mutually independent sampled fields
$\{\phi_{t}\}_{t \in 1:T}$ is available. The empirical structure function
of the field can be computed as follows:
\begin{align}
  D_{\phi}[i]
  &= \begin{cases}
       0 & \text{if $a[i] = 0$,}\\
       \displaystyle \frac{1}{T \times a[i]}
       \sum_{t=1}^{T} b_{t}[i] & \text{else}\\
     \end{cases}\\
  \intertext{with:}
  a[i] &= \alpha\,|\mathcal{R}_i|, \\
  b_{t}[i] &= \alpha\,\sum_{j \in \mathcal{R}_i}
             \bigl(\phi_{t}[j] - \phi_{t}[i+j]\bigr)^{2},
\end{align}
for any $\alpha \not= 0$ and where $\mathcal{R}_i$ is the set of indices $j$
such that $(j,j+i)\in\mathcal{V}^{2}$ and $|\mathcal{R}_i|$ denotes the number
of elements of $\mathcal{R}_i$. Introducing the discrete support $S$ such that:
\begin{align}
  S[i]
  &= \begin{cases}
       \eta & \text{if $i \in \mathcal{V}$,}\\
       0 & \text{else,}\\
     \end{cases}
\end{align}
with $\eta \not= 0$, the terms $a[i]$ and $b_{t}[i]$ can be computed by:
\begin{align}
  \label{eq:naive_a}
  a[i] &= \sum\nolimits_{j} S[j]\,S[j+i],\\
  \label{eq:naive_b}
  b_{t}[i] &= \sum\nolimits_{j} S[j]\,S[j+i]\,
             \bigl(\phi_{t}[j] - \phi_{t}[i+j]\bigr)^{2},
\end{align}
where each sum $\sum\nolimits_{j}$ is for all indices $j$ valid for the
subsequent expression. This corresponds to having $\alpha = \eta^{2}$.
Note that indices $i$ and $j$ may be multi-dimensional (as, for example,
\texttt{CartesianIndex} in Julia).

If the sampled fields $\phi_{t}$ have $N$ elements, computing the empirical
structure function takes $\mathcal{O}(T\,N^{2})$ operations which is
prohibitive except for very small $N$.

To accelerate computations, we first note that it is clear from
Eq.~\eqref{eq:naive_a} that the term $a$ is the discrete auto-correlation of
the discrete support. Indeed $a = S \otimes S$ where the discrete correlation
operator $\otimes$ is defined by:
\begin{align}
  \label{eq:discrete_correlation}
  (x \otimes y)[i] = \sum\nolimits_{j} x[j]\,y[j+i]\,
\end{align}
which can be quickly computed by means of Fast discrete Fourier transforms
(FFTs):
\begin{align}
  \label{eq:fast_discrete_correlation}
  x \otimes y = \mathtt{fftshift}(\mathtt{ifft}(
  \mathtt{fft}(\mathtt{zeropad}(x, \mathtt{dims}))^{*} \times
  \mathtt{fft}(\mathtt{zeropad}(y, \mathtt{dims})))),
\end{align}
where $\mathtt{zeropad}(z, \mathtt{dims})$ yields array $z$ extended to
dimensions $\mathtt{dims}$ which must be chosen large enough to avoid aliasing
and to be suitable for the FFT\footnote{FFTs are more efficient when dimensions
  are multiple of small prime numbers}, $\mathtt{fft}$ yields the FFT of its
argument, superscript $*$ denotes element-wise complex conjugation, $\times$
denotes element-wise multiplication (Hadamard product), $\mathtt{ifft}$ yields
the inverse\footnote{for real-valued arrays $x$ and $y$ it is assumed that
  $\mathtt{ifft}$ yields the real part of the inverse FFT} FFT of its argument,
and $\mathtt{fftshift}$ performs a circular permutation of its argument so as
to put the zeroth frequency assumed by the FFT in the geometrical center of the
array. In Eq.~\eqref{eq:fast_discrete_correlation}, if $F$ is the linear
operator implementing the FFT, then the inverse FFT is implemented by
$(1/N')\,F^{*}$ with $F^{*}$ the adjoint (conjugate transpose) of $F$ and
$N' = \mathtt{prod}(\mathtt{dims})$ the size of the FFT arguments. For other
definitions of the FFT, the left hand-side of
Eq.~\eqref{eq:fast_discrete_correlation} may have to be multiplied by a factor
that only depends on $N'$.

Hence the term $a$  can be computed as:
\begin{align}
  a
  &= S \otimes S\\
  \label{eq:fast_a}
  &= \mathtt{fftshift}(\mathtt{ifft}(|\widehat{w}_{0}|^{2})), \\
  \intertext{with:}
  \widehat{w}_{0} &= \mathtt{fft}(\mathtt{zeropad}(S, \mathtt{dims})),
\end{align}
where $|z|^{2}$ denotes the element-wise squared modulus of array $z$. If
$N' = \mathtt{prod}(\mathtt{dims})$ is the size of the zero-padded arrays, the
FFT and its inverse takes $\mathcal{O}(N'\,\log N')$ operations. As all other
functions take $\mathcal{O}(N')$ operations, computing the term $a$ by means of
FFTs takes about $\mathcal{O}(2\,N'\,\log N')$ operations which is much faster
than $\mathcal{O}(N^{2})$ by the \emph{naive} method.

To accelerate the computation of the term $b_{t}$, we develop it to rewrite it as a sum of correlation products:
\begin{align}
  b_{t}[i]
  &= \sum\nolimits_{j} S[j]\,S[j+i]\,
    \bigl(\phi_{t}[j] - \phi_{t}[i+j]\bigr)^{2}\notag\\
  &= \sum\nolimits_{j} S[j]\,\phi_{t}[j]^{2}\,S[j+i]
    + \sum\nolimits_{j} S[j]\,S[j+i]\,\phi_{t}[j+i]^{2}
    - 2\,\sum\nolimits_{j} S[j]\,\phi_{t}[j]\,S[j+i]\,\phi_{t}[j+i]\notag\\
\intertext{which shows that:}
  b_{t}
  &= (S \times \phi_{t}^{2}) \otimes S
    + S \otimes (S \times \phi_{t}^{2})
    - 2\,(S \times \phi_{t}) \otimes (S \times \phi_{t})\\
  \label{eq:fast_b}
  &= \mathtt{fftshift}(\mathtt{ifft}(
    2\,\Re(\widehat{w}_{0}^{*} \times \widehat{w}_{2,t})
    - 2\,|\widehat{w}_{1,t}|^{2})
  ), \\
  \intertext{with:}
  \widehat{w}_{1,t}
  &= \mathtt{fft}(\mathtt{zeropad}(S \times \phi_{t}, \mathtt{dims})),\\
  \widehat{w}_{2,t}
  &= \mathtt{fft}(\mathtt{zeropad}(S \times \phi_{t}^{2}, \mathtt{dims})),
\end{align}
where (as before) the power $\phi_{t}^{2}$ and the multiplication $\times$ are
applied element-wise.

The FFT of the $b_{t}$ terms may be summed over $t$ to save $T - 1$ inverse
FFT's. All these lead to the following fast algorithm to compute the empirical
structure function of $T$ sampled random fields $\{\phi_{t}\}_{t\in1:t}$:
\begin{equation}
  \label{alg:fast-struct-func}
  \begin{array}{l}
    \widehat{w}_{0} = \mathtt{fft}(\mathtt{zeropad}(S, \mathtt{dims}))\\
    \widehat{b} \gets 0\\
    \text{for $t \in 1:T$}\\
    \qquad \widehat{w}_{1} \gets
    \mathtt{fft}(\mathtt{zeropad}(S \times \phi_{t}, \mathtt{dims}))\\
    \qquad \widehat{w}_{2} \gets
    \mathtt{fft}(\mathtt{zeropad}(S \times \phi_{t}^{2}, \mathtt{dims}))\\
    \qquad \widehat{b} \gets \widehat{b} +\bigl(
    \Re(\widehat{w}_{0}^{*} \times \widehat{w}_{2})
    - |\widehat{w}_{1}|^{2}\bigr) \\
    a \gets \mathtt{fftshift}(\mathtt{ifft}(|\widehat{w}_{0}|^{2}))\\
    b \gets (2/T)\,\mathtt{fftshift}(\mathtt{ifft}(\widehat{b}))\\
    \text{for $i \in 1:N'$}\\
    \qquad D_{\phi}[i] = \begin{cases}
       b[i]/a[i] & \text{if $a[i] > \tau$,}\\
       0 & \text{else}\\
     \end{cases}\\
  \end{array}
\end{equation}
where $\tau > 0$ is a threshold to account for rounding errors when $a$ is
computed by means of FFTs. When $S$ takes its values on $\{0,1\}$,
$a[i] = |\mathcal{R}_{i}|$ and the entries of $a$ should be integers, so we
take $\tau = 1/2$ to disentangle between $a[i] \simeq 0$ and $a[i] \gtrsim 1$
as would be done by rounding $a$ to the nearest integer. Noting that
multiplying $S$ by any non-zero factor cancels and thus has no incidence on the
computed empirical structure function, the fast algorithm may also be applied
to any discrete support $S$ with nonnegative entries (this includes the
\emph{normalized support} introduced in Eq.~\eqref{eq:S-def}) but the value of
$\tau$ must be then be chosen appropriately. The number of operations in the
fast algorithm in Eq.~\eqref{alg:fast-struct-func} is dominated by the
$2\,T + 3$ FFTs.

In the $\mathtt{StructureFunctions}$\footnote{\url{https://github.com/emmt/StructureFunctions.jl}} Julia package, an instance $A$ of
$\mathtt{EmpiricalStructureFunction}$
may be used as follows:
\begin{align*}
  &A = \mathtt{EmpiricalStructFunc}(S)
  &&\text{\# create structure function object with support $S$}\\
  &\mathtt{for}\ \phi_{t}\ \mathtt{in}\ \Phi
  && \text{\# integrate structure function}\\
  &\qquad\mathtt{push!}(A, \phi_{t})\\
  &\mathtt{end}\\
  &T = \mathtt{nobs}(A)
  && \text{\# get the number of observations}\\
  &D_{\phi} = \mathtt{values}(A)
  && \text{\# compute the structure function}
\intertext{At any moment:}
  &a = A\mathtt{.den}
  &&\text{\# yields the denominator $a = S \otimes S$}\\
  &\widehat{b} =  A\mathtt{.num}
  && \text{\# yields the numerator in the frequency domain}
\end{align*}
Note that the denominator is computed at creation of the structure and that its
values are already soft-thresholded.

\end{document}
