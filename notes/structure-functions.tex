\documentclass{article}

\usepackage[a4paper,margin=25mm]{geometry}
\usepackage[colorlinks=true,unicode=true]{hyperref}
\usepackage{amsmath}
\usepackage{amssymb}
%\usepackage{amsfonts} % automatically loaded by amsmath and amssymb
\usepackage{mathtools}
\DeclareMathOperator{\Var}{Var}
\DeclareMathOperator{\Cov}{Cov}

\begin{document}
\title{Structure Function and Covariance}
\author {Éric Thiébaut}
\date{1st version: March 10, 2023; last revision: July 3, 2023}
\maketitle

\abstract{This note describes how to express the covariance $\Cov_{\varphi}$ of
  a Gaussian random field $\varphi \sim \mathcal{N}(0, \Cov_{\varphi})$ given
  the structure function $D_{\varphi}$ of $\varphi$ and the standard deviation
  $\sigma$ of the \emph{piston} over a finite support $\mathcal{V}$. The
  covariance $\Cov_{\varphi}$ is invertible if $\sigma \not= 0$.}

\section{Structure function of a random field}

The structure function of the random field $\phi(r)$ is defined by:
\begin{equation}
  \label{eq:D-phi}
  D_{\phi}(\Delta r) = \langle(\phi(r) - \phi(r+\Delta r))^{2}\rangle
\end{equation}
which holds at a given time whatever $r$ (stationarity) and where
$\langle\ldots\rangle$ denotes expectation. An example of such a field is the
random phase delays caused by a turbulent medium.

Since Eq.~\eqref{eq:D-phi} holds whatever $r$, the structure function is an
even function:
\begin{equation}
  \label{eq:D-phi-even}
  D_{\phi}(-\Delta r) =  D_{\phi}(\Delta r).
\end{equation}

Developping the expression of the field structure function and assuming that
$\phi(r)$ is centered $\forall r$, i.e.\ $\langle\phi(r)\rangle = 0$, yields:
\begin{align}
  D_{\phi}(\Delta r)
  %&= \langle(\phi(r) - \phi(r+\Delta r))^{2}\rangle \notag \\
  &= \langle\phi(r)^{2}\rangle + \langle\phi(r+\Delta r)^{2}\rangle
    - 2\,\langle\phi(r)\,\phi(r+\Delta r)\rangle\notag \\
  &= \Var_{\phi}(r) + \Var_{\phi}(r + \Delta r) - 2 \Cov_{\phi}(r, r + \Delta r)
\end{align}
where $\Cov_{\phi}(r,r')$ is the spatial covariance of the field given by:
\begin{align}
  \Cov_{\phi}(r,r')
  &= \langle\phi(r)\,\phi(r')\rangle
\end{align}
while $\Var_{\phi}(r) = \Cov_{\phi}(r,r)$ is the spatial variance of the field.

Hence the spatial covariance of the field can be expressed as:
\begin{equation}
  \label{eq:Cov-phi}
  \Cov_{\phi}(r,r') = \frac12\,\bigl[
    \Var_{\phi}(r) + \Var_{\phi}(r') - D_{\phi}(r - r')
  \bigr].
\end{equation}


\section{De-Pistoned field}

The de-pistoned field $\psi(r)$ over a finite size volume $\mathcal{V}$ is
defined by ($\forall r \in \mathcal{V}$):
\begin{align}
  \psi(r) = \phi(r) - \frac{1}{|\mathcal{V}|} \int_{\mathcal{V}} \phi(r')\,\mathrm{d}r',
\end{align}
with $|\mathcal{V}|=\int_{\mathcal{V}}\mathrm{d}r$ the volume of $\mathcal{V}$.

Introducing the scaled support function\footnote{so-called scaled \emph{pupil
    function} in optics} $S(r)$ such that for any function $f(r)$:
\begin{align}
  \label{eq:S-def}
  \int f(r)\,S(r)\,\mathrm{d}r = \frac{1}{|\mathcal{V}|} \int_{\mathcal{V}} f(r)\,\mathrm{d}r,
\end{align}
the de-pistoned field writes:
\begin{align}
  \psi(r) = \phi(r) - \int \phi(r')\,S(r')\,\mathrm{d}r'.
\end{align}

A property of the scaled support function is that:
\begin{align}
  \int S(r)\,\mathrm{d}r = 1,
\end{align}
which directly follows from Eq.~\eqref{eq:S-def} by taking $f(r) = 1$ whatever $r$.

Obviously, the structure functions of the field $\phi(r)$ and of the
de-pistoned field $\psi(r)$ are the same, except for the restriction that
$\Delta r$ must fit inside $\mathcal{V}$ for the structure function of the
de-pistoned field.

By linearity and since the field is centered and the aperture finite, the
de-pistoned field is also centered.
\begin{quote}
  \emph{Proof:}
  \begin{align}
    \langle\psi(r)\rangle
    = \langle\phi(r)\rangle -
      \frac{1}{|\mathcal{V}|} \int_{\mathcal{V}} \langle\phi(r')\rangle\,\mathrm{d}r'
    = 0
  \end{align}
  which follows from the linearity of the sum and of the integral and from that
  $\langle\phi(r)\rangle = 0$ whatever $r$. $\blacksquare$
\end{quote}

The covariance of the de-pistoned field can be computed as follows
($\forall (r,r') \in \mathcal{V}^{2}$):
\begin{align}
  \Cov_{\psi}(r,r')
  &= \langle\psi(r)\,\psi(r')\rangle \notag \\
  &= \left\langle\left(\phi(r) - \int \phi(r'')\,S(r'')\,\mathrm{d}r''\right)\,
    \left(\phi(r') - \int \phi(r'')\,S(r'')\,\mathrm{d}r''\right)\right\rangle \notag \\
  &= \Cov_{\phi}(r,r')
    - \int \Cov_{\phi}(r,r'')\,S(r'')\,\mathrm{d}r''
    - \int \Cov_{\phi}(r',r'')\,S(r'')\,\mathrm{d}r'' \notag\\
  &\quad + \iint \Cov_{\phi}(r'',r''')\,S(r'')\,\mathrm{d}r''\,S(r''')\,\mathrm{d}r'''.
    \label{eq:Cov-psi-1}
\end{align}
From Eq.~\eqref{eq:Cov-phi}:
\begin{align}
  \int \Cov_{\phi}(r,r')\,S(r')\,\mathrm{d}r'
  &= \frac{1}{2}\,\int \left(
    \Var_{\phi}(r) + \Var_{\phi}(r') - D_{\phi}(r - r')
    \right)\,S(r')\,\mathrm{d}r'\notag\\
  &= \frac{1}{2}\,\Var_{\phi}(r)
    + \frac{1}{2}\,\int \left(
    \Var_{\phi}(r') - D_{\phi}(r - r')\right)\,S(r')\,\mathrm{d}r',
    \label{eq:Cov-psi-2}
\end{align}
and averaging over the aperture one more time yields:
\begin{align}
  \iint \Cov_{\phi}(r,r')\,S(r)\,\mathrm{d}r\,S(r')\,\mathrm{d}r'
  &= \int \Var_{\phi}(r)\,S(r)\,\mathrm{d}r
  - \frac{1}{2}\,\iint D_{\phi}(r - r')\,S(r)\,\mathrm{d}r\,S(r')\,\mathrm{d}r'.
    \label{eq:Cov-psi-3}
\end{align}
Putting Eqs.~\eqref{eq:Cov-phi}, \eqref{eq:Cov-psi-2}, and \eqref{eq:Cov-psi-3}
in Eq.~\eqref{eq:Cov-psi-1} yields:
\begin{align}
  \Cov_{\psi}(r,r')
  &= \frac{1}{2} \int \left(D_{\phi}(r - r'') + D_{\phi}(r' - r'')\right)\,S(r'')\,\mathrm{d}r''
    - \frac{1}{2} \, D_{\phi}(r - r') \notag\\
  &\quad - \frac{1}{2}\,\iint D_{\phi}(r'' - r''')\,S(r'')\,\mathrm{d}r''\,S(r''')\,\mathrm{d}r'''.
    \label{eq:Cov-psi-4}
\end{align}
Hence the covariance of the de-pistoned field only depends on the structure
function of the field.

To quickly compute $\Cov_{\psi}(r,r')$, it is advantageous to introduce:
\begin{subequations}
  \begin{equation}
    \label{eq:C-phi}
    V_{\psi}(r) = K_{\phi}(r)
    - \frac{1}{2}\,\int K_{\phi}(r')\,S(r')\,\mathrm{d}r',
  \end{equation}
  with:
  \begin{equation}
    \label{eq:K-phi}
    K_{\phi}(r) = \int D_{\phi}(r - r')\,S(r')\,\mathrm{d}r',
  \end{equation}
\end{subequations}
to rewrite the covariance of the de-pistoned field as:
\begin{equation}
  \boxed{
    \Cov_{\psi}(r,r')
    = \frac{1}{2}\,\left[
      V_{\psi}(r) + V_{\psi}(r') - D_{\phi}(r - r')
    \right].
  }
  \label{eq:Cov-psi}
\end{equation}
Taking $r = r'$ in the above expression straightforwardly shows that
$V_{\psi}(r)$ is the variance $\Var_{\psi}(r)$ of the de-pistoned field $\psi$.
Since the field $\phi$ and the de-pistoned field $\psi$ have the same structure
function Eq.~\eqref{eq:Cov-psi} is the counterpart of Eq.~\eqref{eq:Cov-phi}.


\section{Invertible covariance}

Since the piston of the de-pistoned field $\psi(r)$ is zero, the covariance of
the de-pistoned field is rank deficient and thus non-invertible. Adding a
random piston $\alpha \sim \mathcal{N}(0,\sigma^{2})$ to the de-pistoned field
$\psi(r)$ yields the field ($\forall r \in \mathcal{V}$):
\begin{align}
  \label{eq:varphi-def}
  \varphi(r) = \psi(r) + \alpha,
\end{align}
which has an invertible covariance provided $\sigma \not= 0$. Since $\alpha$
and $\psi(r)$ are mutually independent whatever $r$, the covariance of the
field $\varphi(r)$ is trivial to compute:
\begin{align}
  \Cov_{\varphi}(r,r')
  &= \Cov_{\psi}(r,r') + \sigma^{2} \notag\\
  &= \frac{1}{2}\,\left[
      V_{\varphi}(r) + V_{\varphi}(r') - D_{\varphi}(r - r')
    \right],
  \label{eq:Cov-varphi}
\end{align}
where $D_{\varphi} = D_{\phi}$ the structure function of $\varphi$ and with:
\begin{equation}
  \label{eq:Var-varphi}
  V_{\varphi}(r) = \Var_{\varphi}(r) = V_{\psi}(r) + \sigma^{2}
\end{equation}
its variance.

The random field $\varphi(r)$ for $r \in \mathcal{V}$, is a centered Gaussian
field whose covariance $\Cov_{\varphi}$ is given by Eqs.~\eqref{eq:Cov-varphi}
and \eqref{eq:Var-varphi} and only depends on the structure function
$D_{\varphi}$ of the field and on the standard deviation $\sigma$ of the
piston. If $\sigma \not= 0$, the covariance $\Cov_{\varphi}$ is invertible.

\end{document}
